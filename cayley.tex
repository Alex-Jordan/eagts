\begin{chap}{Cayley Graphs}
%
\begin{sect}{Cayley Graphs}
\begin{para}
If $G$ is a group and $C\sbs G$ (i.e., a list of some elements of $G$), then the 
command produces the Cayley graph $X(G,C)$:
\end{para}
%
\begin{sagecode}
\begin{sageinput}
G = DihedralGroup(5)
C = [G.gen(0), G.gen(1)]
CG = G.cayley_graph( generators = C)
CG.show()  # not tested
\end{sageinput}
\end{sagecode}
%
\begin{para}
The output is a directed graph, even if $C$ is inverse-closed; in the latter
case we can convert $G$ to a graph by
\end{para}
%
\begin{sagecode}
\begin{sageinput}
CGU = CG.to_undirected()
CGU.show()  # not tested
\end{sageinput}
\begin{sageoutput}
\end{sageoutput}
\end{sagecode}
%
\begin{para}
Note that $C$ is not required to generate $G$. There are a number of keywords
to this command (for example, \texttt{side} and \texttt{simple}) but I cannot
follow their documentation and so I do not know what they do.
\end{para}
%
\begin{para}
Note that the \texttt{Graph()} command makes it easy to construct Cayley
graphs ourselves. For example, a \define{cubelike graph}{cubelike} is a Cayley
graph for $GF(2)^d$. We can construct them using \texttt{Graph()}.
\end{para}
%
\begin{sagecode}
\begin{sageinput}
def cubelike(d, vector_list):
    VS = VectorSpace(GF(2),d)
    vxs = range(2^d)
    return Graph([vxs, lambda i,j: VS[i]-VS[j] in vector_list])
\end{sageinput}
\end{sagecode}
%
\begin{para}
Here \verb|vector_list| is a list of 01-vectors of length $d$
For a general group, we can use the following to create an undirected Cayley graph.
\end{para}
%
\begin{sagecode}
\begin{sageinput}
G = DihedralGroup(5)
C = [G.gen(0), G.gen(0)^-1, G.gen(1), G.gen(1)^-1]
CG = Graph( [ G.list(), lambda g, h: h*g^(-1) in C])
CG.show()  # not tested
\end{sageinput}
\begin{sageoutput}
\end{sageoutput}
\end{sagecode}
%
\begin{para}
This will produce an incorrect result if $C$ is not inverse-closed.
\end{para}
%
\begin{para}
Sage provides access to GAP, and hence access to any group that is
in GAP or can be constructed in GAP.  Be warned that the documentation
for GAP probably outweighs that for Sage.
\end{para}
%
\end{sect}
%
\begin{sect}{The Higman-Sims Graph}
%
\begin{para}
We construct the famous Higman-Sims graph as a Cayley graph for a permutation
group. Our code is based on an interesting paper by Jorgensen and Klin
from the Electronic Journal of Combinatorics
(\url{http://www.combinatorics.org/Volume_10/PDF/v10i1r17.pdf}).
They provide constructions for a number of srg's on 100 vertices.
\end{para}
%
\begin{para}
First we form a permutation group.
\end{para}
%
\begin{sagecode}
\begin{sageinput}
H1 = PermutationGroup([[(1,2,3,4,5)], [(6,7,8,9,10)], [(2,3,5,4)]])
H1
\end{sageinput}
\begin{sageoutput}
Permutation Group with generators [(6,7,8,9,10), (2,3,5,4), (1,2,3,4,5)]
\end{sageoutput}
\end{sagecode}
%
\begin{para}
The argument to \texttt{PermutationGroup()} is a list where each item in the
list is a list of cycles. Using \verb|H1.order()| shows that \verb|H1| has 
order 100 while \verb|H1.gens()| provides a list of generators---naturally they 
are the three elements we used, but in a different order.
The command \verb|H1.cayley_graph()| will return a directed Cayley graph using
the elements of \verb|H.gens()| as the connection set.
\end{para}
%
\begin{para}
Obviously, to get the Higman-Sims graph we need a special connection set.
For this we make use of the following set of 22 triples:
\end{para}
%
\begin{sagecode}
\begin{sageinput}
ds = [(1,0,0), (4,0,0), (0,1,1), (0,4,1), (2,0,1), (4,2,1),\
  (4,3,1), (0,1,2), (0,4,2), (1,2,2), (1,3,2), (2,0,2),\
  (3,1,2), (3,2,2), (3,3,2), (3,4,2), (4,0,2), (0,1,3),\
  (0,4,3), (1,0,3), (2,2,3), (2,3,3)]
\end{sageinput}
\end{sagecode}
%
\begin{para}
We set
\end{para}
%
\begin{sagecode}
\begin{sageinput}
y,z,x = H1.gens()
\end{sageinput}
\end{sagecode}
%
\begin{para}
and then create a connection set
\end{para}
%
\begin{sagecode}
\begin{sageinput}
C = [ x^it[0]*y^it[1]*z^it[2] for it in ds]
C
\end{sageinput}
\begin{sageoutput}
[(1,2,3,4,5), (1,5,4,3,2), (2,3,5,4)(6,7,8,9,10), 
(2,3,5,4)(6,10,9,8,7), (1,5,3,4), (1,4,5,2)(6,8,10,7,9), 
(1,4,5,2)(6,9,7,10,8), (2,5)(3,4)(6,7,8,9,10), 
(2,5)(3,4)(6,10,9,8,7), (1,5)(2,4)(6,8,10,7,9), 
(1,5)(2,4)(6,9,7,10,8), (1,4)(2,3), (1,3)(4,5)(6,7,8,9,10), 
(1,3)(4,5)(6,8,10,7,9), (1,3)(4,5)(6,9,7,10,8), 
(1,3)(4,5)(6,10,9,8,7), (1,2)(3,5), (2,4,5,3)(6,7,8,9,10), 
(2,4,5,3)(6,10,9,8,7), (1,4,3,5), (1,2,5,4)(6,8,10,7,9), 
(1,2,5,4)(6,9,7,10,8)]
\end{sageoutput}
\end{sagecode}
%
\begin{para}
Here $x$, $y$ and $z$ are our original generators and $C$ is a subset
of $G$ consisting of elements of the form $x^iy^jz^k$. So they are
specified by triples $(i,j,k)$ and \verb|ds| is indeed a set of triples.
We perform a partial check on our work by testing if $C$ is inverse-closed.
\end{para}
%
\begin{sagecode}
\begin{sageinput}
all( [x^-1 in C for x in C] )
\end{sageinput}
\begin{sageoutput}
True
\end{sageoutput}
\end{sagecode}
%
\begin{para}
The graph we want is
\end{para}
%
\begin{sagecode}
\begin{sageinput}
G = H1.cayley_graph( generators=C)
\end{sageinput}
\end{sagecode}
%
\begin{para}
How can we confirm that this is the Higman-Sims graph?  Well, it is connected,
regular, and has exactly three eigenvalues (as we see in the factored characteristic polynomial):
\end{para}
%
\begin{sagecode}
\begin{sageinput}
G.is_connected(), G.is_regular(), G.am().fcp()
\end{sageinput}
\begin{sageoutput}
(True, True, (x - 22) * (x + 8)^22 * (x - 2)^77)
\end{sageoutput}
\end{sagecode}
%
\begin{para}
Since the Higman-Sims graph is determined by its eigenvalues, we have it.
We used \verb|G.am().fcp()| to get the factored characteristic polynomial
of $G$, because it's shorter than \verb|G.characteristic_polynomial().factor()|
even with tab completion. Also, for a graph on any significant number of vertices
there is very little to be gained by looking at the coefficients of its
characteristic polynomial---they are far too large!
\end{para}
%
\begin{para}
We can use \verb|G.girth()| to see that $G$ is triangle-free and we can see that $G$ is vertex transitive.
\end{para}
%
\begin{sagecode}
\begin{sageinput}
G.girth(), G.is_vertex_transitive()
\end{sageinput}
\begin{sageoutput}
(4, True)
\end{sageoutput}
\end{sagecode}
%
\begin{para}
To verify that $G$ is arc-transitive, we prove more by showing
that the stabilizer of a vertex has exactly three orbits on vertices.
We set things up with
\end{para}
%
\begin{sagecode}
\begin{sageinput}
G.relabel()
part = [[0], [1..99]]
orbs = G.automorphism_group( return_group=False,\
    partition=part, orbits=True)
\end{sageinput}
\end{sagecode}
%
\begin{para}
Some explanations are in order. As created the vertices of $G$ are permutations,
after \verb|G.relabel()| they are the integers in $[0..99]$. Now
\verb|part| is a simple partition of $G$, isolating the vertex $0$.  Then 
\verb|orbs| is the list of orbits of the subgroup of automorphisms of $G$
that fix each cell of the partition \texttt{part}, effectively the 
stabiliser of the vertex $0$. From
\end{para}
%
\begin{sagecode}
\begin{sageinput}
len(orbs)
\end{sageinput}
\begin{sageoutput}
3
\end{sageoutput}
\end{sagecode}
%
\begin{para}
we infer that the vertex stabiliser has three orbits, which must be 0,
the neighbors of 0 and the vertices at distance two from 0. We conclude
that $\aut G$ is a rank-three permutation group. Its order?
\end{para}
%
\begin{sagecode}
\begin{sageinput}
G.automorphism_group(return_group=False,order=True)
\end{sageinput}
\begin{sageoutput}
88704000
\end{sageoutput}
\end{sagecode}
%
\begin{para}
The Higman-Sims graph contains many triangle-free srg's as subgraphs.
For example, the subgraphs obtained by deleting a vertex and its neighbors
or by deleting two adjacent vertices and its neighbors. Its vertices
can be partitioned into two copies of the Hoffman-Singleton graph,
but that's another story.
\end{para}
%
\end{sect}
%
\begin{sect}{The 600-Cell}
%
\begin{para}
It's a wikifact that if $f:=(1+\sqrt{5})/2$, then the vertices of a 600-cell 
centered at the origin of 4-space with edges of length $1/f$ 
can be given as follows: 
\begin{enumerate}
    \begin{listitem}
    16 vertices of the form
    \[
        \frac12(\pm1,\pm1,\pm1,\pm1),
    \]
    \end{listitem}
    \begin{listitem}
    8 vertices obtained all permutations of the coordinates of
    \[
        (\pm1,0,0,0)
    \]
    \end{listitem}
    \begin{listitem}
    96 vertices obtained by taking all even permutations of
    \[
        \frac12\left(\pm1,\pm f,\pm f^{-1},0\right).
    \]
    \end{listitem}
\end{enumerate}  
\end{para}
%
\begin{para}
The first step is produce a $120\times4$ matrix $U$ with these vectors
its rows. The game is to do this without using a for-loop.
We set up our background
\end{para}
%
\begin{sagecode}
\begin{sageinput}
x = QQ['x'].0
K.<f> = NumberField(x^2-x-1)
RR4 = VectorSpace(K,4) 
H.<i,j,k> = QuaternionAlgebra(K,-1,-1)
\end{sageinput}
\end{sagecode}
%
\begin{para}
So we work with quaternions over $\rats(f)$, where $f$ is a zero of $f^2-f-1$.
Due to an unfortunate feature of Python, we must remember now \textbf{never}
to use $i$ or $j$ or $k$ as an index in a list comprehension.  If we evaluate 
\verb|[i^2 for i in [1..5]]| the value of $i$ becomes $5$ once complete.
\end{para}
%
\begin{para}
Now we create $U$. 
\end{para}
%
\begin{sagecode}
\begin{sageinput}
CP3 = CartesianProduct([-1,1],[-1,1],[-1,1],[0])
CP4 = CartesianProduct([-1,1],[-1,1],[-1,1],[-1,1])
fv1 = RR4([1/2,f/2,(f-1)/2,0])
fv2 = RR4([(f-1)/2,1/2,f/2,0])
fv3 = RR4([f/2,(f-1)/2,1/2,0])
xx= [fv1.pairwise_product(RR4(sign_v))\
    for sign_v in CP3]\
    +[fv2.pairwise_product(RR4(sign_v))\
    for sign_v in CP3]\
    +[fv3.pairwise_product(RR4(sign_v))\
    for sign_v in CP3]
mm = Matrix(xx)
mma = mm.matrix_from_columns([1,0,3,2])
mmb = mm.matrix_from_columns([2,3,0,1])
mmc = mm.matrix_from_columns([3,2,1,0])
MM = block_matrix([mm,mma,mmb,mmc], nrows=4)
ww = RR4([1/2,1/2,1/2,1/2])
xxd =\ 
    [ww.pairwise_product(RR4(sign_v))\
    for sign_v in CP4]+[RR4([1,0,0,0]),\  
    RR4([-1,0,0,0]),RR4([0,1,0,0]),RR4([0,-1,0,0]),\ 
    RR4([0,0,1,0]), RR4([0,0,-1,0]),RR4([0,0,0,1]),\
    RR4([0,0,0,-1])]
U = MM.stack(Matrix(xxd))
\end{sageinput}
\end{sagecode}
%
\begin{para}
The best we can say for the above code is that it works. Interestingly
enough, the columns of $U$ are orthogonal since $U^TU =30I$.
\end{para}
%
\begin{sagecode}
\begin{sageinput}
U.transpose()*U
\end{sageinput}
\begin{sageoutput}
[30  0  0  0]
[ 0 30  0  0]
[ 0  0 30  0]
[ 0  0  0 30]
\end{sageoutput}
\end{sagecode}
%
\begin{para}
Two vectors are adjacent in the 1-skeleton if their inner product is $1/f$.
\end{para}
%
\begin{sagecode}
\begin{sageinput}
DC = Graph([[1..120], lambda i,j: U[i-1].inner_product(U[j-1]) == 1/2*f])
\end{sageinput}
\end{sagecode}
%
\begin{sagecode}
\begin{sageinput}
DC.is_regular(), DC.is_vertex_transitive(), DC.degree()[0], DC.diameter()
\end{sageinput}
\begin{sageoutput}
(True, True, 12, 5)
\end{sageoutput}
\end{sagecode}
%
\begin{para}
We compute the orbits of the vertex stabilizer of $\aut{DC}$.
\end{para}
%
\begin{sagecode}
\begin{sageinput}
DCgrp = DC.automorphism_group( partition = [[1],[2..120]])
map( len, DCgrp.orbits())
\end{sageinput}
\begin{sageoutput}
[12, 12, 20, 20, 12, 12, 1, 30, 1]
\end{sageoutput}
\end{sagecode}
%
\begin{para}
We convert the rows of $U$ to quaternions.
\end{para}
%
\begin{sagecode}
\begin{sageinput}
qu = [uu[0]+uu[1]*i+uu[2]*j +uu[3]*k for uu in U.rows()]
\end{sageinput}
\end{sagecode}
%
\begin{para}
The reduced trace of a quaternion $q$ is $q+q^*$, in other words it is
twice its real part. The multiplicative order of a quaternion is determined
by its real trace. We compute the partition of $qu$ by reduced trace
\end{para}
%
\begin{sagecode}
\begin{sageinput}
def ls_dict( tuples):
    dc = {} 
    for pair in tuples:
        if dc.has_key( pair[0]):
            dc[ pair[0]].append(pair[1])
        else:
            dc[ pair[0]] = [pair[1]]
    return dc
\end{sageinput}
\end{sagecode}
%
\begin{sagecode}
\begin{sageinput}
tuples = [(qq.reduced_trace(),qq) for qq in qu]
dc = ls_dict( tuples)
[(kit, len(dc[kit])) for kit in dc.keys()]
\end{sageinput}
\begin{sageoutput}
[(0, 30), (1, 20), (2, 1), (-f, 12), (-f + 1, 12), 
(f - 1, 12), (f, 12), (-2, 1), (-1, 20)]
\end{sageoutput}
\end{sagecode}
%
\begin{para}
Comparing the number of quaternions with given reduced trace with the
sizes of the orbits of the vertex stabilizer, suggests that the two partitions
are equal. You should verify this. You should also verify that the
multiplicative order of an element of \verb|qu| is determined by its 
reduced trace. (Use \verb|qq.order()|.) In particular the quaternions
with reduced trace $f$ or$1-f$ have order 10. The quaternions in
\verb|qu| form a multiplicative group isomorphic to $SL(2,5)$.
You can access this group in Sage by
\end{para}
%
\begin{sagecode}
\begin{sageinput}
T = SL(2,5)
T
\end{sageinput}
\begin{sageoutput}
Special Linear Group of degree 2 over Finite Field of size 5
\end{sageoutput}
\end{sagecode}
%
\begin{para}
The 600-cell is a Cayley graph for this group with a connection
set consisting of a conjugacy class of elements of order 10.
There are two such conjugacy classes, one formed by the elements
with reduced trace $f$ and the other consisting of the elements of reduced
trace $1-f$. We show now that the Cayley graph with respect to the first of these
conjugacy classes is isomorphic to $DC$, you should verify that the second is.
\end{para}
%
\begin{sagecode}
\begin{sageinput}
rtf = [it for it in qu if it.reduced_trace() == f]
conn = Set([ it^(-1)*rtf[0]*it for it in qu]).list()
CG = Graph( [qu, lambda q1, q2: q2*q1^(-1) in conn])
\end{sageinput}
\end{sagecode}
%
\begin{para}
and now
\end{para}
%
\begin{sagecode}
\begin{sageinput}
CG.is_isomorphic( DC)
\end{sageinput}
\begin{sageoutput}
True
\end{sageoutput}
\end{sagecode}
%
\begin{para}
The quaternion
\[
    th = (-1/2) + 1/2*i + 1/2*j + 1/2*k
\]
has order three and its orbits on $DC$ are cocliques. The quotient
over its orbits is a graph on 40 vertices which, like \verb|DC|, is locally
an icosahedron. We confirm the first claim:
\end{para}
%
\begin{sagecode}
\begin{sageinput}
nbhd = DC.subgraph( vertices=DC[1])
nbhd.is_isomorphic( graphs.IcosahedralGraph())
\end{sageinput}
\begin{sageoutput}
True
\end{sageoutput}
\end{sagecode}
%
\end{sect}
%
\begin{sect}{Cubelike Graphs}
%
\begin{para}
A cubelike graph is a Cayley graph for $\ints_2^d$. The connection
set of such a graph can be encoded by a $d\times m$ matrix with distinct
columns. If $M$ is such a matrix then we can get a list of its columns with \verb|cols = M.columns()| and we can recover $M$ by \verb|M = Matrix(cols).transpose()|.
\end{para}
%
\begin{para}
The natural choice for the vertices of a cubelike graph are the elements of
\end{para}
%
\begin{sagecode}
\begin{sageinput}
V = VectorSpace( GF(2)), d)  # not tested
\end{sageinput}
\end{sagecode}
%
\begin{para}
but these are not hashable. We can make a vector \verb|v| hashable by
\end{para}
%
\begin{sagecode}
\begin{sageinput}
v.set_immutable() # not tested
\end{sageinput}
\end{sagecode}
%
\begin{para}
and use vectors as vertices, or work as follows.
\end{para}
%
\begin{sagecode}
\begin{sageinput}
G = Graph( [[0..len(vls)-1],\
    lambda i,j: vls[i]-vls[j] in V.list()]) # not tested
\end{sageinput}
\end{sagecode}
%
\begin{sagecode}
\begin{sageinput}
P = graphs.PetersenGraph()
D = P.incidence_matrix()
B = D.change_ring( GF(2))  # convert to a matrix over GF(2)
B0 = B.submatrix( nrows=B.nrows()-1)  # delete last row
B0
\end{sageinput}
\begin{sageoutput}
[1 1 1 0 0 0 0 0 0 0 0 0 0 0 0]
[0 0 1 1 1 0 0 0 0 0 0 0 0 0 0]
[0 0 0 0 1 1 1 0 0 0 0 0 0 0 0]
[0 0 0 0 0 0 1 1 1 0 0 0 0 0 0]
[0 1 0 0 0 0 0 0 1 1 0 0 0 0 0]
[1 0 0 0 0 0 0 0 0 0 1 1 0 0 0]
[0 0 0 1 0 0 0 0 0 0 0 0 1 1 0]
[0 0 0 0 0 1 0 0 0 0 0 1 0 0 1]
[0 0 0 0 0 0 0 1 0 0 1 0 0 1 0]
\end{sageoutput}
\end{sagecode}
%
\begin{para}
For humans it can be convenient to encode binary vectors of length $d$ as 
integers between 0 and $2^{d-1}$. With $G$ as just defined, its connection
set will be the correct set of integers.
\end{para}
%
\begin{sagecode}
\begin{sageinput}
def cubelike(vecls):
    d = len(vecls[0])
    VS = VectorSpace(GF(2),d)
    return Graph([[0..(2^d-1)],\
     lambda i,j: VS[i]-VS[j] in vecls])
\end{sageinput}
\end{sagecode}
%
\end{sect}
%
\end{chap}
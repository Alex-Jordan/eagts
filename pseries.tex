%!TEX root = Practice.tex

\chapter{Polynomials and Power Series}

\section{Forbidden Words}

Let $\al$ denote a binary string, and let $L$ denote
the set of binary of binary strings that do not contain $\al$ as a substring. 
How many strings are there in $L$ with length $n$?

Let $M$ denote the binary
strings that contain exactly one copy of $\al$, as a suffix. 
If $\eps$ denotes the empty string, then
\begin{equation}
    \label{eq:eL01}
    \eps+L(0+1) = L+M.
\end{equation}
Also $L\al$ clearly contains $M$ as a subset, but for example if $\al=111$,
then
\[
    L\al =M+M1+M11.
\]
We develop a precise description of $L\al$. Let $\al\backslash\be$ denote the 
set of suffixes $\be_1$ such that
$\be_0\be_1=\be$ and $\be_0$ is a suffix of $\al$. For example if
\[
    \al =100110,\quad \be=1011,
\]
then
\[
    \al\backslash\be = \{11\},\quad \al\backslash\al = \{\eps,0110\}
\]

\begin{lemma}
    \[
        L\al = M\al\backslash\al.\qed
    \]
\end{lemma}

Note that $\al\backslash\al$ is a finite language, i.e., a finite set of strings.

If $N$ is a set of binary strings, let $N(t)$ denote its generating series.
This is what we get when we substitute $t$ for $0$ and $1$, a string of length
$\ell$ maps to $t^\ell$ and the language maps to its generating series.
Now \eqref{eq:eL01} yields
\[
    1+2t L(t) = L(t)+M(t)
\]
and, if $D=\al\backslash\al$ and $|\al|$ denotes the length of $\al$, 
the lemma yields that
\[
    L(t)t^{|\al|} = M(t) D(t).
\]

\begin{lemma}
    \[
        L(t) =\left(1-2t+\frac{t^{|\al|}}{D(t)}\right)^{\!-1}.
    \]
\end{lemma}

Let's use this to compute some numbers. We first set up the ring of formal
power series over the rationals.

\begin{sageblock}
    Rt.<t> = QQ[[]]
\end{sageblock}

If $\al=111$, then $D(t)=1+t+t^2$ and 
\[
    L(t) = \sage{((1-2*t+ t^3/(1+t+t^2))^(-1)).O(9)}
\]

As an exercise, write a program which takes a binary string $\al$ and
computes the generating series for $\al\backslash\al$. (The string arrives
as a python string, there is no reason why your procedure should not work
for strings over an arbitrary alphabet.)

The square root of the series $L(2t)$ has non-negative integer coefficients.
(Use \verb|L.subs(t=2*t).sqrt()|, where \verb|L| is the series.)
Does this hold for other forbidden substrings?

For a second variant, suppose $\cF=\{\seq\al1d\}$ is a set of binary strings. 
Let $L$ be the set of strings that contain no element of $\cF$ and let $M_i$ 
denote the set of strings that contain one copy of $\al_i$, as a suffix, bit do not
contain any other copies of strings in $\cF$. If $|\cF|=2$, we have
equations
\begin{align*}
    \eps+L(0+1) &= L +M_1 +M_2\\
    L\al_1 &= M_1(\al_1\backslash\al_1) +M_2(\al_1\backslash\al_2)\\
    L\al_2 &= M_1(\al_2\backslash\al_1) +M_2(\al_2\backslash\al_2).
\end{align*}
Write a program that, given $\al_1$ and $\al_2$, computes the generating
series for $L$, $M_1$ and $M_2$.


\section{Fractions and Series}

Many generating series can be expressed as rational functions, and 
such expressions provide useful information. We illustrate this.
First we create a ring of polynomials, and the corresponding field
of fractions. (We will not need the latter, as it happens.)

\begin{sageblock}
    Qu.<u> = QQ[]
    F = FractionField( Qu)
\end{sageblock}

Now we can work with our series $L(t)$ from the previous section.

\begin{sageexample}
sage: rf = (1-2*u+ u^3/(1+u+u^2))^(-1)
sage: rf.denominator()
sage: rf.numerator()
sage: rf.subs( u=1/2)
sage: rf.subs( u=t).O(9)
\end{sageexample}

Note that \verb|t| is the generator of \verb|Rt|, which is $\rats[[t]]$.
So we have used \verb|subs()| to convert a rational function to
a power series.

We can also compute partial fraction decompositions. For these
to be useful the zeros of the denominator should belong to the field
we are working over.

\begin{sageexample}
sage: Cs = FractionField( PolynomialRing( CDF, 's'))
sage: whole, parts = rf.subs(u=Cs.gen()).partial_fraction_decomposition() 
sage: len( parts)
sage: parts[0]
sage: parts[1]
sage: parts[2]  
\end{sageexample}

\section{Solving Equations}

If $C(t)$ denotes the generating series for the Catalan numbers, then
\[
    C(t) = 1+tC(t)^2
\]
From this it follows that
\[
    C(t) = \frac{1}{2t}(1-\sqrt{(1-4t)}).
\]
We can implement this as follows:
\begin{sageblock}
    Ct = (1/2)*(1-(1-4*t).sqrt()).shift(-1)
\end{sageblock}
The first coefficients are what we expect:
\begin{sageexample}
sage: Ct.O(9)
\end{sageexample}

If
\[
    \Phi(u) := 1+tu^2
\]
then our equation for $C(x)$ about may be written as
\[
    C(t) = \Phi(C(t));
\]
hence $C(t)$ is defined as a fixed point of a map on $\rats[[t]]$.
We can solve this directly. Recursively define a sequence of series $C_i(t)$ 
by $C_0(t)=1$ and $C_{n+1}(t) =\Phi(C_n(t))$.
We then find that
\begin{align*}
    C_1(t) &= 1 + t\\
    C_2(t) &= 1 + t + 2t^2 + t^3\\
    C_3(t) &= 1 + t + 2t^{2} + 5t^{3} + 6t^{4} + 6t^{5} + 4t^{6} + t^{7}.
\end{align*}
and we see that the coefficients of $C_n(t)$ are accurate up to 
(and including) degree $n$. (At least for $n\le3$, but it's easy to 
verify this holds for larger values of $n$.) One problem
here is that if we continue in this vein, our expression for $C_k(t)$
will have $2^k$ terms, of which only the first $k+1$ are sure to be accurate.
The following code avoids this extra work.
\begin{sageblock}
def iter( Fun, init, acc, n):
    val = init
    for i in [1..n]:
        val = Fun( val).O(acc)
        acc += 1
\end{sageblock}

As an exercise, verify that if the first $k+1$ coefficients of $C_k(t)$
are correct, then the first $k+2$ coefficients of $C_{k+1}(t)$ are correct.


\section{Newton-Raphson}

We can find a solution to our equation $C(t) =\Phi(C(t))$ more efficiently by
using Newton-Raphson. Note that $\Phi$ is a power series (actually polynomial in $u$)
and
\[
    \Phi(C_0+\de) \approx \Phi(C_0) +\Phi'(C_0)\de;
\]
this works provided $\de$ is divisible by a power of $t$.
So if $C_0$ is an approximate solution to $\Phi(C(t))=C(t)$, then our aim is to 
choose $\de$ so that
\[
    C_0 + \de = \Phi(C_0) +\Phi'(C_0)\de,
\]
which implies that
\[
    \de = \frac{\Phi(C_0)-C_0}{1-\Phi'(C_0)}.
\]
Hence our proposed update formula is now
\[
    C_{n+1} = C_n + \frac{\Phi(C_n)-C_n}{1-\Phi'(C_n)}
\]
Here
\[
    \Phi'(C_n(t)) = 2tC_n(t).
\]

We can run a quick test. With
\begin{sageexample}
sage: phi = lambda u: 1+t*u^2
sage: nr = lambda u: u + (phi(u)-u)/(1-2*t*u)
\end{sageexample}
we find that the first 22 terms of \verb|nr(nr(nr(1+t)))| are correct.

Your task is to write an efficient implementation of this method.


\section{Counting Trees}

Our aim is to derive the generating series for the number of trees on
$n$ vertices (or, more precisely, the number of isomorphism classes of
tress on $n$ vertices). We use $R(x)$ to denote the generating series
for rooted trees on $n$ vertices. We can check that
\[
    R(x) = x+x^2+2x^2+\ldots
\]
The problem is to get further. We do not have an explicit formula for
$R(x)$, just an equation:
\[
    R(x) = x\exp\left(\sum_{k\ge1}\frac{1}{k}R(x^k)\right).
\]
We use the technology from the previous sections to solve this.

We see that $R(x)$ is a fixed point of a map of the form
\[
    R = \Phi(R)
\]
and we can use this to compute $R$.

We set up our ring of power series:
\begin{sageblock}
    QQx.<x> = QQ[[]]
\end{sageblock}
If \verb|R| is a power series $\sum r_n x^n$, then
\begin{verbatim}
    R.V(i)
\end{verbatim}
denotes the series $\sum r_n x^{ni}$. So our mapping on series is easy.
\begin{sageblock}
 Phi = lambda R, n: x*(sum([R.V(i)/i for i in [1..n]]).exp(n+1))
 def iter2( init, acc, times):
     val = Phi( init, acc)
     for i in [1..times]:
         acc += 1
         val = Phi( val, acc)
     return val
\end{sageblock}
The result of \verb|iter2( x+x^2,2,7)| is:
\[
    \sage{latex( iter2( x+x^2,2,7))}
\]

You should prove that this procedure produces a sequence of
series that converges to the generating series for rooted trees.
You will probably find this procedure converges somewhat more
quickly then your convergence proof implies.

Next we apply the Newton-Raphson procedure.
Here's the code, without explanation :-(
\begin{sageblock}
def nrtrees(n):
    Qx.<x> =QQ[[]]
    rr = Qx([0,1,1])
    lim = 2
    for j in [1..n]:
        ss = x*(sum([rr.V(i)/i for i in [1..2*lim]]).exp(2*lim+1))
        lim = 2*lim+1
        rr = rr +(ss-rr)/(1-ss)
        rr = Qx(rr.list())
    return rr
\end{sageblock}
Note that \verb|rr = Qx([0,1,1])| is an up-market version of \verb|rr = x+x^2|.
The line \verb|rr = Qx(rr.list())| is a hack to force sage to work with the
required degree of precision. Your exercise is to show that the output of this 
procedure is correct.

If $T(x)$ is the generating series for trees, then
\[
    T(x) = R(x) -\frac12(R(x)^2-R(x^2))
\]
Hence we can easily get the number of trees from $R(x)$.

\begin{sageexample}
sage: rr = nrtrees(3)
sage: rr - (1/2)*(rr^2-rr.V(2)).O(14)
\end{sageexample}
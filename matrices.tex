\begin{chap}{Matrices}
%
\begin{sect}{Continuous Quantum Walks}
%
\begin{para}
One goal of this section is to introduce you to the joys of numerical linear
algebra with sage.
\end{para}
%
\begin{para}
Suppose $A$ is the adjacency matrix of $X$. We define $H(t)=H_X(t)$ by
\[
    H =\exp(iAt)
\]
We are concerned with the absolute values of the entries $H(t)$, and in particular
want to know when we can have
\[
    |H(\tau)_{u,v}| =1
\]
for some time $\tau$ and some vertices $u$ and $v$. Since $A$ is symmetric 
it has a spectral decomposition
\[
    A =\sum_\th \th E_\th
\]
and consequently
\[
    H(t) = \sum_\th \exp(i\th t)E_\th
\]
\end{para}
%
\begin{para}
To use this formula we need an orthogonal basis for each eigenspace.
The difficulty if that if we work in floating point, then we have
to decide which eigenvalues are actually equal before we determine
an eigenspace.
\end{para}
%
\begin{para}
We are going to use scipy and numpy. 
It seems that numpy is basically an array package, which 
provides such things as $k$-dimensional arrays. (Computer scientists seem
to think these are useful, I have no idea why.) Scipy builds on numpy, providing
access to standard linear algebra routines (eigenthings, svd, etc.)
There is on-line documentation for scipy and numpy, but it is not very good.
(For example it illustrates how solve linear equations by inverting the matrix 
of coefficients.) The syntax for scipy and numpy is not consistent with sage, or 
with python. 
\end{para}
%
\begin{para}
Anyway, here is the opening incantation:
\end{para}
%
\begin{sageblock}
import scipy
from scipy import linalg as lin
\end{sageblock}
%
\begin{para}
We will go through some simple computations, so you can get a feel for things.
\end{para}
%
\begin{verbatim}
    P = graphs.PetersenGraph()
    AP = A.am()
\end{verbatim}
%
\begin{para}
Now with
\end{para}
%
\begin{verbatim}
    la,v = linalg.eigh(AP)
\end{verbatim}
%
\begin{para}
we get the eigenvalues of $AP$ (in $\la$) and the eigenvectors (in \(v\)).
The routine \verb|linalg.eigh| takes a Hermitian matrix and returns real eigenvalues.
Eigenvectors corresponding to distinct eigenvalues are orthogonal. Note that the
eigenvectors are returned by
\verb|linalg.eigh| as numpy arrays, and there is a argument for
wrapping it in a routine that immediately transforms them into sage vectors.
\end{para}
%
\begin{para}
We intend to make use of the fact that if the vector $\seq z1m$ are an orthogonal basis
for a subspace then the matrix representing projection onto the subspace is
\[
    z_1z_1^T+\cdots+z_mz_m^T
\]
The problem is that our eigenvalues have been computed in floating point, and
so we need a routine with arguments $\la$ and $\eps$ which divides the
eigenvalues into equivalence classes, where two eigenvalues are equivalent if their
difference is less than $\eps$. The output of the routine
is a dictionary keyed on distinct eigenvalues and the entry with given eigenvalue
is the set of indices of the eigenvalues equivalent to the key.
\end{para}
%
\begin{para}
Given this it is straightforward to write the code that takes this dictionary and
returns a list of pairs consisting of an eigenvalue and its associated projection.
\end{para}
%
\end{sect}
%
\end{chap}
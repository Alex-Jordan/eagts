\begin{chap}{Getting Started}
    
\begin{sect}{Getting Information}
%
\begin{para}
The Sage tutorial at \url{http://www.sagemath.org/doc/tutorial/index.html}
is the best starting point, and is very easy to read.  The "constructions" manual 
is of very little use for graphs. The Sage reference manual at
\url{http://www.sagemath.org/doc/reference/index.html} is vast. You will eventually 
need to read the sections on graphs, matrices, vector spaces, finite fields,
power series and commutative rings.
\end{para}
%
\begin{para}
If you are not up to speed with python, the first four or five chapters of 
"Dive into Python"  (\url{http://diveintopython.org/})
could be useful and are available on line.  (I do not think
you will get much out the later chapters dealing with XML and HTML.)
I also recommend the Python Tutorial (\url{http://docs.python.org/tutorial/}).  
You need to become comfortable with lists, dictionaries and list comprehensions.
You can get quite a ways in Sage without knowing anything much about objects 
and classes, but you will need some familiarity with Python.
\end{para}
%
\begin{para}
There are two basic ways of running Sage.  One is to start Sage in a
terminal window and then type \texttt{notebook()}.  This starts up a 
web browser and you work from there.
\end{para}
%
\begin{para}
The alternative, which I use, is to start Sage in a terminal window and then 
open up an editor. Set the working directory in Sage to the directory where your 
files are. (Sage recognizes \texttt{ls}, \texttt{cd} and \texttt{pwd)}---so don't use
\texttt{ls} as the name of a list!) Write your code in the editor, save it as
\texttt{stuff.sage} (say), then type  \texttt{ load stuff.sage} in your terminal window. 
Now you can run your code in Sage. You can load as many files as you like, and if 
you edit a file, load it again.
\end{para}
%
\begin{para}
You can load a file with the command \texttt{attach stuff.sage} and then
Sage will automatically reload your file each time
you save it.  A disadvantage of this approach is that certain syntax errors 
in your file can confuse Sage considerably.  In this case you may have to 
hit control-C (and perhaps wait a bit) to unlock things.  I have found it more
convenient just to use the load command, as above.
\end{para}
%
\begin{para}
You can save data to a file:
\end{para}
%
\begin{sagecode}
\begin{sageinput}
some_list = ['A', 356, x^2]
save( some_list, 'smlst')
\end{sageinput}
\end{sagecode}
%
\begin{para}
and reload it with: 
\end{para}
%
\begin{sagecode}
\begin{sageinput}
renamed_list = load('smlst')
renamed_list
\end{sageinput}
\begin{sageoutput}
['A', 356, x^2]
\end{sageoutput}
\end{sagecode}
%
\begin{para}
(The output to save is not human readable.)             
When you're working in Sage, on-line help can be obtained using \texttt{help()} 
and the \texttt{tab} key. Thus typing \texttt{tab} after \texttt{graphs?} or \texttt{graphs??} gives 
information about the command \texttt{graphs()}.  You will find situations where this
does not work.  Tab completion can sometimes save you typing in the full name for
some command, but you will also find there are many contexts where it does not.
\end{para}
%
\begin{para}
Do not forget that the command you are trying to understand may be a Python
command; reasonably enough these are not explained in the Sage docs.
\end{para}
%
\end{sect}
%
\begin{sect}{Basics}
%
\begin{para}
(I'm assuming that you can get Sage running and evaluate the sum 1+1, say.)
Sage comes with many graphs preinstalled.  Thus the command
\end{para}
%
%
\begin{sagecode}
\begin{sageinput}
K = graphs.CompleteGraph(5); K
\end{sageinput}
\begin{sageoutput}
Complete graph: Graph on 5 vertices
\end{sageoutput}
\end{sagecode}
%
\begin{para}
sets $K$ equal to the complete graph on 5 vertices.  
Now \verb|K.show()| produces a drawing of the graph in a separate window. 
\end{para}
%
\begin{sagecode}
\begin{sageinput}
K.show()   # not tested
\end{sageinput}
\end{sagecode}
%
\begin{para}
The command
\end{para}
%
\begin{sagecode}
\begin{sageinput}
K.vertices()
\end{sageinput}
\begin{sageoutput}
[0, 1, 2, 3, 4]
\end{sageoutput}
\end{sagecode}
%
\begin{para}
displays the vertices of our graph and
\end{para}
%
\begin{sagecode}
\begin{sageinput}
K.edges()
\end{sageinput}
\begin{sageoutput}
[(0, 1, None), (0, 2, None), (0, 3, None), (0, 4, None), 
(1, 2, None), (1, 3, None), (1, 4, None), 
(2, 3, None), (2, 4, None), (3, 4, None)]
\end{sageoutput}
\end{sagecode}
%
\begin{para}
displays the edges. To avoid the empty labels, try
\end{para}
%
\begin{sagecode}
\begin{sageinput}
K.edges(labels=False)
\end{sageinput}
\begin{sageoutput}
[(0, 1), (0, 2), (0, 3), (0, 4), (1, 2), 
(1, 3), (1, 4), (2, 3), (2, 4), (3, 4)]
\end{sageoutput}
\end{sagecode}
%
\begin{para}
The command \verb|K.degree()| produces a list of the vertex degrees,
while \verb|K.deg(u)| gives the degree of the vertex $u$. Further 
\verb|K[i]| returns the neighbors of $i$ in $K$. If your graph is
regular (\verb|K.is_regular()| returns \verb|True|) then its valency
is given by
\end{para}
%
\begin{sagecode}
\begin{sageinput}
K.degree()[0]
\end{sageinput}
\begin{sageoutput}
4
\end{sageoutput}
\end{sagecode}
%
\begin{para}
There are many built-in graphs in Sage. Thus:
\end{para}
%
\begin{sagecode}
\begin{sageinput}
C = graphs.CycleGraph(8); C
\end{sageinput}
\begin{sageoutput}
Cycle graph: Graph on 8 vertices
\end{sageoutput}
\end{sagecode}
%
\begin{para}
gives us the cycle on 8 vertices. To get the list of possible graphs,
type \verb|graphs.| at the prompt, followed by a tab. (That's 8 keypresses---6 letters,
a period and a tab.)
\end{para}
%
\begin{para}
There are many graph operations we can use. We can get the complement of
our cycle by
\end{para}
%
\begin{sagecode}
\begin{sageinput}
CC = C.complement(); CC
\end{sageinput}
\begin{sageoutput}
complement(Cycle graph): Graph on 8 vertices
\end{sageoutput}
\end{sagecode}
%
\begin{para}
and its line graph by
\end{para}
%
\begin{sagecode}
\begin{sageinput}
LC = C.line_graph(); LC
\end{sageinput}
\begin{sageoutput}
Graph on 8 vertices
\end{sageoutput}
\end{sagecode}
%
\begin{para}
Of course a cycle is isomorphic to its line graph, which we can verify:
\end{para}
%
\begin{sagecode}
\begin{sageinput}
C.is_isomorphic(LC)
\end{sageinput}
\begin{sageoutput}
True
\end{sageoutput}
\end{sagecode}
%
\begin{para}
Another way to verify that $LC$ is the cycle $C_8$ is to verify that it
is connected
\end{para}
%
\begin{sagecode}
\begin{sageinput}
LC.is_connected()
\end{sageinput}
\begin{sageoutput}
True
\end{sageoutput}
\end{sagecode}
%
\begin{para}
and that it is regular of degree two
\end{para}
%
\begin{sagecode}
\begin{sageinput}
LC.degree()
\end{sageinput}
\begin{sageoutput}
[2, 2, 2, 2, 2, 2, 2, 2]
\end{sageoutput}
\end{sagecode}
%
\begin{para}
Sage supplies the Petersen graph
\end{para}
%
\begin{sagecode}
\begin{sageinput}
P = graphs.PetersenGraph(); P
\end{sageinput}
\begin{sageoutput}
Petersen graph: Graph on 10 vertices
\end{sageoutput}
\end{sagecode}
%
\begin{para}
and we can verify this is correct by computing is diameter, girth
and the number of vertices since these three properties characterize the graph.
\end{para}
%
\begin{sagecode}
\begin{sageinput}
P.diameter(), P.girth(), P.num_verts()
\end{sageinput}
\begin{sageoutput}
(2, 5, 10)   
\end{sageoutput}
\end{sagecode}
%
\begin{para}
In practice it is very important to check that any graph you construct is
the one you wanted.  It may be hard to prove that your graph is correct,
but simple checks can still be useful. One common source of error is an incomplete
understanding of the commands you use.  For example
\end{para}
%
\begin{sagecode}
\begin{sageinput}
K2 = graphs.CompleteGraph(2)
K3 = graphs.CompleteGraph(3)
M = K2.union(K3)
M
\end{sageinput}
\begin{sageoutput}
Graph on 3 vertices
\end{sageoutput}
\end{sagecode}
%
\begin{para}
produces a graph on three vertices!
\end{para}
%
\begin{para}
I should have used \verb|K2.disjoint_union(K3)|.
\end{para}
%
\begin{sagecode}
\begin{sageinput}
K2.disjoint_union(K3)
\end{sageinput}
\begin{sageoutput}
Complete graph disjoint_union Complete graph: Graph on 5 vertices
\end{sageoutput}
\end{sagecode}
%
\end{sect}
%
\begin{sect}{Fiddling with Vertices and Edges}
%
\begin{para}
Although we can get a lot done with the graphs already in Sage, most of the time
we will need to construct our own. The simplest approach is to modify one of 
the prebuilt graphs.
\end{para}
%
\begin{para}
For an examples, we start with the circulant graph $G$ on 9 vertices
with connection set $C=\{1,2,7,8\}$. So $V(G)=\mathbb{Z}_9$ and
vertices $i$ and $j$ are adjacent if their difference is in $C$.
\end{para}
%
\begin{sagecode}
\begin{sageinput}
G = graphs.CirculantGraph( 9, [1,2]); G
\end{sageinput}
\begin{sageoutput}
Circulant graph ([1, 2]): Graph on 9 vertices
\end{sageoutput}
\end{sagecode}
%
\begin{para}
For later use we also make a copy of $G$:
\end{para}
%
\begin{sagecode}
\begin{sageinput}
H = G.copy()
\end{sageinput}
\begin{sageoutput}
\end{sageoutput}
\end{sagecode}
%
\begin{para}
You can use \verb|G.show()| to get a view of your graph, this can provide
a useful check.
\end{para}
%
%
\begin{sagecode}
\begin{sageinput}
G.show()
\end{sageinput}
\begin{sageoutput}
\end{sageoutput}
\end{sagecode}
%
\begin{para}
We find the neighbors of 0:
\end{para}
%
\begin{sagecode}
\begin{sageinput}
G[0]
\end{sageinput}
\begin{sageoutput}
[8, 1, 2, 7]
\end{sageoutput}
\end{sagecode}
%
\begin{para}    
and attempt to confirm that $(0,1)$ is an edge by
\end{para}
%
\begin{sagecode}
\begin{sageinput}
(0,1) in G.edges()
\end{sageinput}
\begin{sageoutput}
False
\end{sageoutput}
\end{sagecode}
%
\begin{para}
and then realize that we should have tried
\end{para}
%
%
\begin{sagecode}
\begin{sageinput}
(0,1) in G.edges(labels=False)
\end{sageinput}
\begin{sageoutput}
True
\end{sageoutput}
\end{sagecode}
%
\begin{para}
or
\end{para}
%
\begin{sagecode}
\begin{sageinput}
G.has_edge(0,1)
\end{sageinput}
\begin{sageoutput}
True
\end{sageoutput}
\end{sagecode}
%
\begin{para}
The last alternative is recommended.
We can delete this edge:
\end{para}
%
\begin{sagecode}
\begin{sageinput}
G.delete_edge(0,1)
\end{sageinput}
\begin{sageoutput}
\end{sageoutput}
\end{sagecode}
%
\begin{para}
and confirm the effect by
\end{para}
%
\begin{sagecode}
\begin{sageinput}
G.degree()
\end{sageinput}
\begin{sageoutput}
[3, 3, 4, 4, 4, 4, 4, 4, 4]
\end{sageoutput}
\end{sagecode}
%
\begin{para}
Note that \verb|G.delete_edge()| alters $G$ in place, which is why I
prepared a copy of it. The copy is unaltered:
\end{para}
%
\begin{sagecode}
\begin{sageinput}
H.degree()
\end{sageinput}
\begin{sageoutput}
[4, 4, 4, 4, 4, 4, 4, 4, 4]
\end{sageoutput}
\end{sagecode}
%
\begin{para}
Deleting an edge that is not present has no effect. We can delete
the members of a list of edges with 
\end{para}
%
\begin{sagecode}
\begin{sageinput}
an_edge_list = [(0, 7), (2, 4)]
G.delete_edges( an_edge_list)
G.edges(labels=False)
\end{sageinput}
\begin{sageoutput}
[(0, 2), (0, 8), (1, 2), (1, 3), 
(1, 8), (2, 3), (3, 4), (3, 5), 
(4, 5), (4, 6), (5, 6), (5, 7), 
(6, 7), (6, 8), (7, 8)]
\end{sageoutput}
\end{sagecode}
%
\begin{para}
Similarly we can add an edge or list of edges---type \verb|G.add|
followed by tab to see the things we can add and \verb|G.delete|
followed by tab for the things we can delete (which of course include
vertices and list of vertices).
\end{para}
%
\begin{para}
Adding a vertex increases the vertex set but, naturally enough, does
not add any edges.
If $S$ is a list of vertices of $G$, we can add a new vertex adjacent to 
each vertex in $S$ as follows.
\end{para}
%
\begin{sagecode}
\begin{sageinput}
G.add_edges( [(10,s) for s in [1,2,3]] )
G.edges(labels=False)
\end{sageinput}
\begin{sageoutput}
[(0, 2), (0, 8), (1, 2), (1, 3), 
(1, 8), (1, 10), (2, 3), (2, 10), 
(3, 4), (3, 5), (3, 10), (4, 5), 
(4, 6), (5, 6), (5, 7), (6, 7), 
(6, 8), (7, 8)]
\end{sageoutput}
\end{sagecode}
%
\begin{para}
Here the argument \verb|[ (10,s) for s in S ]| to \verb|G.add_edges()| 
is a simple example of a list comprehension.  
We will find these are very useful.
\end{para}
%
\begin{para}
Edge contraction is one basic operation in graph theory that is not built 
into Sage.  The following procedure identifies two vertices in a graph.
\end{para}
%
%% indent with spaces, not tabs
\begin{sagecode}
\begin{sageinput}
def contract(H, e):
    G = H.copy() 
    u, v = e[0], e[1]
    vedges = [(v,x) for x in G[u]]  
    G.delete_vertex(u)
    G.add_edges( vedges)
    return G
\end{sageinput}
\begin{sageoutput}
\end{sageoutput}
\end{sagecode}
%
\begin{para}
A useful generalization would be a procedure that took a graph $G$
and a partition of its vertex set as its input, and returned
a graph with the cells of the partition as vertices and where
two cells are adjacent if there is an edge that joins a vertex
in the first cell to a vertex in the second.
\end{para}
%
\end{sect}
%
\begin{sect}{Graph()}
%
\begin{para}
The command \verb|Graph()| provides a very useful way to construct graphs.
To begin, we will invoke it in the form \verb|X = Graph( [vertices, adj_pred] )|.
There is just one argument, a list consisting of the
the vertices is the vertex set of X (itself a list), and a predicate (or function)
\verb|adj_pred| that takes two vertices $u$ and $v$ and returns \verb|True| or
\verb|False| according as they are adjacent or not.
\end{para}
%
\begin{para}
As a trivial example, we construct the path on 9 vertices:
\end{para}
%
\begin{sagecode}
\begin{sageinput}
P9 = Graph([[0..8], lambda i,j: i-j == 1])
P9.edges(labels=False)
\end{sageinput}
\begin{sageoutput}
[(0, 1), (1, 2), (2, 3), (3, 4), (4, 5), (5, 6), (6, 7), (7, 8)]
\end{sageoutput}
\end{sagecode}
%
\begin{para}
Here \verb|lambda| is a synonym for ``function'', its inputs precede the colon
and the subsequent text computes its value. By using \verb|lambda|, we can define
our predicate inline. You should experiment to see what happens if you type 
\verb|=| in place of \verb|==|, since you will eventually do it by accident, 
and it is good to be able to recognize what's gone wrong.
We could have defined the predicate first:
\end{para}
%
\begin{sagecode}
\begin{sageinput}
def adj_pred( i,j):
    return i - j == 1
\end{sageinput}
\begin{sageoutput}
\end{sageoutput}
\end{sagecode}
%
\begin{para}
and then created our graph with
\end{para}
%
\begin{sagecode}
\begin{sageinput}
P9_2 = Graph([[0..8], adj_pred])
P9_2.is_isomorphic(P9)
\end{sageinput}
\begin{sageoutput}
True
\end{sageoutput}
\end{sagecode}
%
\begin{para}
This illustrates that Python allows us to pass a function as an argument.
\end{para}
%
\begin{para}
\verb|Graph()| accepts a variety of arguments, including:
\begin{enumerate}
    \begin{listitem}A list of edges.\end{listitem}
    \begin{listitem}A dictionary of lists.\end{listitem}
    \begin{listitem}An adjacency matrix.\end{listitem}
    \begin{listitem}A graph6 string.\end{listitem}
\end{enumerate}
(For the full list, try \verb|Graph??|.)
\end{para}
%
\begin{para}
What's a graph6 string? All we need to know is that it is a compact
encoding of a graph as an ASCII string.  We can produce that string encoding
the graph as follows
\end{para}
%
\begin{sagecode}
\begin{sageinput}
P9.graph6_string()
\end{sageinput}
\begin{sageoutput}
'HhCGGC@'
\end{sageoutput}
\end{sagecode}
%
\begin{para}
There are other programs (e.g., Brendan McKay's \verb|geng|) that will produce files of
graphs presented as graph6 strings, and so we can read these into
Sage and convert them to Sage graphs. 
\end{para}
%
\begin{para}
Here is a very natural way to create a graph, by supplying a list of edges.
\end{para}
%
\begin{sagecode}
\begin{sageinput}
an_edge_list = [(0,1), (1,2), (2, 3), (3, 0), (1,3)]
G = Graph(an_edge_list)
G.degree()
\end{sageinput}
\begin{sageoutput}
[2, 3, 2, 3]
\end{sageoutput}
\end{sagecode}
%
\end{sect}
%
\begin{sect}{Vertices}
%
\begin{para}
In Sage, the vertices of a graph must be hashable. You can test if this holds 
by applying the function \verb|hash()| to whatever you propose to use as a vertex.
If it is not hashable you'll get an error message informing you of the fact, if it
is hashable you'll get back an integer.
\end{para}
%
\begin{para}
Occasionally the experimental approach to determining hashability will be 
unsatisfactory, so I offer some comments.
\end{para}
%
\begin{para}
The output of \verb|vector()| and \verb|matrix()| is not hashable, but can
be made so by using \verb|a.set_immutable()| (where \verb|a| may be a matrix
or a vector). The identity matrix is hashable, as you will find out the first 
time you try to change an entry in it. Subspaces of vector
spaces over finite fields are hashable.
\end{para}
%
\begin{para}
Sage has two classes of sets: \verb|set()| constructs the Python version and
\verb|Set()| the Sage version. Sage \verb|Set|s are hashable, Python \verb|set|s are not.
The available operations differ (go figure).
\end{para}
%
\begin{para}
If \verb|vset| is a list of things you want to use as vertices, and they are not 
readily converted to something hashable, you can construct a graph as follows:
\end{para}
%
\begin{para}
\verb|G = Graph( [[1..len(vset)], lambda i,j: adj_pred( vset[i],vset[j])])|
\end{para}
%
\begin{para}
Here \verb|adj_pred| takes two elements of \verb|vset| and returns
\verb|True| or \verb|False| according as its two arguments are adjacent or not.
\end{para}
%
\begin{para}
Note that many graph operations will go faster when the vertices are integers,
and we can always arrange this after the fact by using \verb|G.relabel()|.
\end{para}
%
\end{sect}
%
\end{chap}

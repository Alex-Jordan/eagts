\begin{chap}{Distance-Regular Graphs}
%
\begin{sect}{Distance Regularity}
%
\begin{para}
First some code to allow us to test if a graph is distance regular.
The main use of this is to provide a test that some graph we have constructed is
distance regular.
\end{para}
%
\begin{sageblock}
def distance_partition( X, vxi):
    xvs = X.vertices()
    E = dict()
    for vxj in xvs:
        dj = X.distance(vxi, vxj)
        E.setdefault(dj,[]).append(vxj)
    return E.values()    
\end{sageblock}
%    
\begin{para}
If $X$ is distance-regular, then the distance partition is equitable
and the quotient matrix with respect to this partition is the same for all
vertices. 
\end{para}
%
\begin{para}
We use the Kneser graphs as a test case.
\end{para}
%
\begin{sageblock}
def kneser( v, k):
    combs = map(set, Combinations([0..v-1],k))
    return Graph( [[0..binomial(v,k)-1],\
     lambda i, j: combs[i].isdisjoint(combs[j])])
\end{sageblock}
%
\begin{para}
We have used python sets here, because they admit the function \verb|is_disjoint()|.
Otherwise we would replace \verb|a.isdisjoint(b)| by
\end{para}
%
\begin{verbatim}
    len( a.intersection(b)) == 0
\end{verbatim}
%
\begin{sageexample}
sage: X = kneser(7,3)
sage: dp = distance_partition( X, X.vertices()[0])
sage: X.is_equitable( dp, quotient_matrix=True) 
\end{sageexample}
%
\begin{para}
If the partition is not equitable, the response would be a simple
\verb|False|.
\end{para}
%
\begin{para}
Many distance-regular graphs are distance-transitive. We can test for this
by verifying that our graph is vertex transitive and the number of
orbits of the stabilizer of a vertex is one less than the diameter.
\end{para}
%
\begin{sageexample}
sage: orbs = X.automorphism_group( return_group=0, orbits=1)
sage: len(orbs) == 1 
sage: pn = [ [X.vertices()[0]], X.vertices()[1:]]
sage: grp = X.automorphism_group( partition=pn)
sage: X.diameter()
sage: len( grp.orbits())
\end{sageexample}
%
\begin{para}
The Kneser graph has a perfect 1-code---a set of seven vertices pairwise at distance
three. If we delete the vertices in a perfect 1-code, the resulting graph
is the Coxeter graph which is a distance-regular cubic graph.
It is not too hard to see that a perfect 1-code will be a coclique of size seven
that is maximal under inclusion. So we can find one and construct the
Coxeter graph as follows.
\end{para}
%
\begin{sageblock}
    colqs = X.complement().cliques_maximal()
    colqs7 = filter( lambda it: len(it)==7, colqs)
    Y = X.copy()
    Y.delete_vertices( colqs7[0])
\end{sageblock}
%
\begin{para}
We confirm that $Y$ has girth seven, you might verify that $Y$ is distance-transitive.
\end{para}
%
\begin{sageexample}
sage: X.girth()
sage: Y.girth()
\end{sageexample}
%
\end{sect}
%
\begin{sect}{Generalized Quadrangles}
%
\begin{para}
A \textsl{generalized quadrangle} is an incidence structure of points
and lines such that
\end{para}
%
\begin{enumerate}
    \begin{listitem}
    Any two points lie on at most one line.
    \end{listitem}
    \begin{listitem}
    If the point $x$ is off the line $\ell$, then there is a unique
    point on $\ell$ collinear with $x$.
    \end{listitem}
\end{enumerate}
%
\begin{para}
This axioms are self-dual, and therefore the incidence structure
dual to a $GQ$ is again a $GQ$.
\end{para}
%
\begin{para}
The smallest interesting example has as points the 15 edges of $K_6$ and
as lines the 15 1-factors of $K_6$; each edge is incident with the three
1-factors that contain it. A $GQ$ is \textsl{thick} if each point is on at least 
three lines and each line contains at least three points. If a $GQ$ is thick then
it is point and line regular, this means there are integers $s$ and $t$ such that
each point lies on exactly $t+1$ lines and each line contains exactly $s+1$
points. Our example above is a $GQ(2,2)$, traditionally denoted by $W(2)$.
\end{para}
%
\begin{para}
Associated to an incidence structure we have a point graph, a line graph
and an incidence graph.  For $W(2)$ the point and line graphs are isomorphic
to $L(K_6)$ (which is very easy to check); the incidence graph is
a bipartite cubic graph on 30 vertices with diameter four and girth eight.
It is known as \textsl{Tutte's 8-cage}. For a thick $GQ$, both the point
and line graphs are strongly regular.
\end{para}
%
\begin{para}
We describe how to construct a family of $GQ$'s denoted by $W(q)$, where
$q$ is a primes power. We will accompany the description with the
code for $W(3)$.
\end{para}
%
\begin{para}
Let $V$ be the vector space of dimension four over $GF(q)$. Its 1-dimensional
subspaces will be the points of our generalized quadrangle. To define
the lines we want a non-degenerate alternating form on $V$; this is
given by an invertible matrix $S$ with diagonal entries zero such that $S+S^T=0$.
(So if $q$ is odd, then $S$ is skew symmetric; if $q$ is even it's symmetric
with zero diagonal.) A subspace $U$ of $V$ is \textsl{isotropic}
if
\[
    u^THv = 0
\]
for all $u$, $v$ in $U$. All 1-dimensional subspaces of $V$ are isotropic and the
lines of $W(q)$ will be the 2-dimensional isotropic subspaces.
\end{para}
%
\begin{para}
Time for some actual work. We define our form:
\end{para}
%
\begin{sageblock}
def beta(u,v):
    return u[0]*v[2]+u[1]*v[3] -u[2]*v[0] -u[3]*v[1]
\end{sageblock}
%
\begin{para}
and create our points and lines:
\end{para}
%
\begin{sageblock}
    V = VectorSpace( GF(3), 4)
    Points = [u[1] for u in V.subspaces(1)]
    Lines = [sub for sub in V.subspaces(2)\
      if beta(sub.matrix()[0],sub.matrix()[1])==0]
\end{sageblock}
%
\begin{para}
Two points $u$ and $v$ are collinear if $\beta(u,v)=0$. Two lines $L$ and $M$ 
are incident if
\end{para}
%
\begin{verbatim}
    (L != M) and (L.intersection(M) != V.zero_subspace())
\end{verbatim}
%
\begin{para}
or if they are not equal and
\end{para}
%
\begin{verbatim}
    det( L.matrix().stack(M.matrix)) == 0
\end{verbatim}
%
\begin{para}
Elements of our vector space $V$ are "mutable", so not hashable, 
and therefore cannot be used as vertices of a graph. This is easily circumvented:
\end{para}
%
\begin{sageblock}
W3 = Graph([range(len(Points)),\
   lambda i,j: beta(Points[i],Points[j])==0], loops=False)
\end{sageblock}
%
\begin{para}
We can check that $W3$ is connected and regular, and that it has exactly three
eigenvalues:
\end{para}
%
\begin{sageexample}
    sage: W3.am().fcp()
\end{sageexample}
%
\begin{para}
The lines of the $GQ$ correspond to the cliques of maximal size, which we
can find by
\end{para}
%
\begin{verbatim}
    cliques = W3.cliques_maximum()
\end{verbatim}
%
\begin{para}
We get the point graph of the dual $GQ$ by
\end{para}
%
\begin{sageblock}
W3d = Graph( [range(len(Lines)), lambda i,j: i != j\
    and det( Lines[i].matrix().stack(Lines[j].matrix())) == 0])
\end{sageblock}
%
\begin{para}
As expected this is not isomorphic to $W3$, but it is strongly regular with
the same parameters.
\end{para}
%
\end{sect}
%
\begin{sect}{The McLaughlin Graph}
%
\begin{para}
A partition $(S,\comp{S})$ of $V(G)$ determines a bipartite subgraph of $G$;
if we replace this subgraph by its bipartite complement, we say that the
resulting graph is got by \textsl{switching} on $S$ (or on $\comp{S}$).
If $S$ is the neighborhood of a vertex $u$ in $G$, then after switching
on $S$ the vertex $u$ is isolated.
\end{para}
%
\begin{sageblock}
def switch(G,sub):
    H = G.copy()
    rest = Set(H.vertices()).difference(Set(sub))
    for u in sub:
        for v in rest:
            if H.has_edge(u,v): H.delete_edge(u,v)
            else: H.add_edge(u,v)
    return H   
\end{sageblock}
%
\begin{para}
We construct a graph on the blocks of the $4$-$(23,7,1)$ design
formed by the words of weight seven in the binary Golay code of length 23.
(This graph is strongly regular.)
We then add 23 vertices $\{0,\ldots,23\}$ corresponding to the 23 coordinate 
positions of a code word, and join the $i$-th new vertex to the blocks
that do not contain it. The result graph is not regular but if we switch on
one of the new vertices and then delete it we get the McLaughlin graph
on 275 vertices, which is strongly regular.
\end{para}
%
\begin{sageblock}
    C = ExtendedBinaryGolayCode()
    D = C.punctured([0])
    words = [ Set(it.support()) for it in D if hamming_weight(it)==7]
    MG = Graph( [words, lambda a,b: len(a.intersection(b))==1])
    MM = MG.copy()
    MM.add_vertices([0..22])
    edges = [ (i,a) for i in [0..22] for a in words if i not in a]
    MM.add_edges( edges)
    McL = switch( MM, MM[0])
    McL.delete_vertex(0)
\end{sageblock}
%
\begin{para}
It's strongly regular, with a big automorphism group:
\end{para}
%
\begin{sageexample}
sage: McL.is_connected()
sage: McL.is_regular()
sage: McL.am().fcp()
sage: McL.automorphism_group(return_group=0,order=1)
\end{sageexample}
%
\begin{para}
The McLaughlin graph gives rise to a regular two-graph. For details on regular
two-graphs, see any recent book on algebraic graph theory. 
\end{para}
%
\begin{sageblock}
    G = McL.copy()
    G.relabel()
    G.add_vertex()
    A = G.am()
    n = G.num_verts()
    J = Matrix(n,n,n^2*[1])
    B = J -1 -A
    C = block_matrix( [A,B,B,A])
    SMcL = Graph( C)
    grp = SMcL.automorphism_group()
\end{sageblock}
%
\begin{para}
and now
\end{para}
%
\begin{sageexample}
sage: grp.order()
\end{sageexample}
\begin{para}
The group is one of Conway's simple groups.
\end{para}
%
\end{sect}
%
\begin{sect}{Drackns: Generating Parameter Sets}
%
\begin{para}
We are interested in distance-regular antipodal covers of $K_n$.
First we explain `cover'. We construct \textsl{cover} of
index $r$ of a graph $X$ (or an \textsl{$r$-fold cover})
as follows. Choose a function $f$ from the arcs (ordered pairs of adjacent
vertices of $X$) to the symmetric group $\sym r$, such that
\[
    F((v,u)) =f((u,v))^{-1}
\]
The vertex set of the cover $X^f$ is 
\[
    V(X)\times \{1,\ldots,r\}
\]
and $(u,i)\sim (v,j)$ if $j = i^{f(u,v)}$. Less formally, we replace each
vertex of $V(X)$ by a coclique of size $r$, and cocliques corresponding
to adjacent vertices are joined by a matching of size $r$. The $r$-cocliques
are called the \textsl{fibers} of the cover.
\end{para}
%
\begin{para}
A cover $X^f$ is \textsl{antipodal} if its diameter is $d$ and two vertices
are in the same fiber if and only they are at distance $d$. The 3-cube is an antipodal
cover of $K_4$ with index two and diameter three. The line graph of the Petersen
graph is an antipodal cover of $K_5$ with index and diameter three.
\end{para}
%
\begin{para}
Here we are concerned with distance-regular antipodal covers of complete graphs $K_n$.
Such covers must have diameter three. There are four obvious parameters $(n,r,a_1,c_2)$,
although these are not independent, in fact counting edges joining neighbors
of $u$ to vertices at distance two from $u$, we get
\[
    n(n-2-a_2) = n(r-1)c_2
\]
whence
\begin{equation}
\label{eq:n1rc2}
    n-1-rc_2 = a_1-c_2.
\end{equation}
The difference $a_1-c_2$ occurs frequently, and we denote it by $\de$.
It is not too hard to show that an antipodal
cover of $K_n$ is distance regular if its diameter is three and any two 
distinct non-adjacent vertices have $c_2$ common neighbors.
\end{para}
%
\begin{para}
Our basic problem is to determine the parameter triples $(n,r,c_2)$
for which a cover exists. This is an impossible problem, so we settle
for generating parameter sets for which there is a good chance that a
cover exists. Here our first design question surfaces: do we want to order
our parameters by $n$ and then $r$ (lexicographically), or by $nr$ then $r$
(for example). We arbitrarily select the first approach.
\end{para}
%
\begin{para}
As a first step we aim to generate a list of quadruples $(n,r,a_1,c_2)$.
Now we know that
\begin{equation}
\label{eq:rbnd}
    2\le r\le n-1.
\end{equation}
What about $c_2$. The neighborhood of a vertex induces a regular graph
on $n-1$ vertices with valency $a_1$, whence $na_1$ must be even.
From (\ref{eq:n1rc2}) we have 
\[
    n-1-a_1 = (r-1)c_2,
\]
from which we infer that if $n$ is odd and $r$ is even, then $c_2$ must be
even. We also have
\begin{equation}
\label{c2bnd}
    1\le c_2 \le \left\lfloor\frac{n-1}{r-1}\right\rfloor
\end{equation}
\end{para}
%
\begin{sageblock}
    def nrc( n):
        lim = (n-1)/(r-1)
        return [(n,r,c) for r in [2..n-1]\ 
         for c in [1..lim] and n*(r-1)*c % 2 == 0]
    def get_a( n, r, c):
        return n, r, n-1-(r-1)*c, c
\end{sageblock}
%
\begin{para}
So we will generate a sequence of 4-tuples, and filter out the ones
that obviously do not work. The most useful condition rests on the formulas
for the multiplicities of the eigenvalues. As a distance-regular graph
with diameter three, a drackn has exactly four distinct eigenvalues
\[
    n > \th > -1 > \tau
\]
where $n$ is simple, $-1$ has multiplicity $n-1$ and $-\th\tau=n-1$.
The eigenvalues $\th$ and $\tau$ are the zeros of
\[
    t^2 -\de t - (n-1).
\]
It is their multiplicities that interest us. With an obvious notation we have
\begin{align*}
    rn &= 1 + (n-1) +m_\th+m_\tau\\
    0  &= (n-1) +(n-1)(-1) +m_\th\th+m_\tau\tau
\end{align*}
and therefore
\begin{equation}
\label{eq:mult}
    m_\th = \frac{(r-1)n(-\tau)}{\th -\tau}.
\end{equation}
\end{para}
%
\begin{para}
Two cases arise. First, if $\th$ and $\tau$ are not integers, then
\[
    \th=\sqrt{n-1},\quad \tau=-\sqrt{n-1}
\]
and their multiplicities are equal to $n(r-1)/2$. Otherwise they are integers
and consequently the discriminant
\begin{equation}
\label{eq:discr}
    (\th-\tau)^2 = \de^2+4(n-1) = (a_1-c_2)^2+4(n-1)
\end{equation}
must be a perfect square. (Note that $\de=n-1-rc_2$.) If this is a perfect square
we can determine $\th$ and $\tau$, and then check that our formula for $m_th$
is an integer.
\end{para}
%
\begin{sageblock}
    def mult_chk( n, r, a, c):
        disc = (a-c)^2+4*(n-1)
        if is_square(disc):
            sqroot = sqrt( disc)
            theta = (a-c+sqroot)/2
            tau = (a-c-sqroot)/2
            mth = n*(r-1)*theta/sqroot
            mtau = n*(r-1)-mth
            return (theta, tau, mth, mtau)
        else:
            return False
\end{sageblock}

\begin{para}
How do we generate 4-tuples? The simplest approach is to use \verb|CartesianProduct|,
but a more efficient strategy is to use generators.
\end{para}
%
\begin{para}
Now we can generate a selected family of 4-tuples. There are other conditions
that must hold, we write a predicate for each of these and then write
a function that takes a 4-tuple and applies each term in a list of
predicates. Note that, when a parameter set is eliminated we need to know
at least one of the predicates that it fails.
\end{para}
%
\begin{para}
There is a very strong case to be made that a better parameterization is available.
The idea is to search on triples
\[
    (-\tau,\theta, r)
\]
Our approach above throws out most triples because $(a_1-c_2)^2+4(n-1)$ is not
a perfect square. Our alternative approach only produces triples where
this condition is satisfied. We need to treat separately the case where
$m_\th=m_\tau$, (equivalently $\de=0$) but this makes sense combinatorially. 
If we want the parameters for covers of $K_n$ with index $r$, then we filter
through the pairs $(-\tau, \theta)$ where $\theta$ runs over the elements
of \verb|divisors(n-1)| such that the Krein bound holds.
\end{para}
%
\end{sect}
%
\end{chap}